\chapter{INTRODUCTION}
The main reason to preserve forests are: to protect the world's most rich ecosystem, which has 90\% of all terrestrial biodiversity; to decrease carbon dioxide gas emission; atmospheric pollution absorption; soil fertilization; influence in the water cycle; wood and biomass stockage \cite{Brooks2006, Gullison2007, Barlow2004}.

The task of monitoring forests is usually done with satellites. Since 1988, it has shown increased effectiveness\cite{Douglas2010a, Santilli2005, Monteiro2008}. This technique allows a deft surveying, in comparison with forest inventory. Forest inventories usually takes years to be concluded, needing more professionals to do the job, as it increases proportionately to the number of forest sampled.

Deforestation alerts are sent to authorities, in intervals of 8 to 15 days, for regions fewer than 12.5\%\footnote{Area of Amazônia Legal monitored by DETER and PRODES} of the whole forest extension. More precisely deforestation data requires more time to compile the results, that are about 12 months.

In a similar way that occurs with forest inventories, that professionals have survey the whole required area, with only few of them taking long time to complete the job. It happens to satellite monitoring systems, when it needs more accurate data or expands the monitoring area, the amount of data to be analyzed adds up too. The number of professionals required to evaluate satellite images increases,and computational processing needed for data compilation, raising the time and costs to extract results.

An alternative, with low cost, that has been adopted more often to handle scientific problems like this one, is to make use of volunteers for data evaluation. This approach is known as Citizen Science \cite{Irwin1995}. Volunteers can collaborate with scientific projects, giving away computational or cognitive power.

This study will evaluate techniques to integrate background modules responsible for whole the satellite image acquisition until the published results, becoming more reliable and having a better performance. In addition, it will increment the current architecture with the possibility of in-situ data collection, providing reability for the final results.
%\sout{This study will evaluate alternative ways to get the most of the volunteers, using a citizen science project.} And will estimate different manners to integrate satellite images acquisition modules, using citizen science as well, to create new areas would ease off, fast and scalable.

This proposal is organized as follows: Chapter \ref{rainforests:sec:intro}, gives a review on the forests importance, known examples of monitoring projects and their work. Chapter \ref{citizen_science:sec:intro}, introduces citizen science projects, their categories and some of the most succeeded examples. The citizen science project, cited earlier to perform deforestation watching, is on chapter \ref{forest_watchers}. At last, the chapter \ref{projectplan:intro} discuss the objectives of this proposal and timetable.



%%There are well known reasons to preserve forests, the main ones are to protect the biodiversity existence, the most rich of the ecosystem with 90\% of all terrestrial biodiversity; $\textrm{CO}_2$ gases emission reduction; atmospheric pollution absorption; soil fertilization; its biggest influence in the water cycle; wood and biomass stockage.
%%
%%To monitor forests, usually uses satellites. Satellites monitoring system has been applied since 1988, and has shown increased effectiveness. This technique allows a deft monitoring, compared to forest inventory. Those usually take years to be concluded, needing a variety of professionals to do the job, having a proportional complexity to forest extensions being sampled.
%%
%%Deforestation alerts to authorities, may be sent among 8 to 15 days of the cause, for regions a bit less than 60\% of the throughout forest extension. More precisely deforestation data requires more time to compile the results, that are about 12 months.
%%
%%In similar way that occurs with the forest inventory when a big extension needs to be sampled, happens to the satellite monitoring system as well, when a more accurate data is needed, bigger resolution or keep monitoring of different territorial extensions. Despite of the increased numbers of professionals required to evaluate satellite images, there is a computational processing increase needed to compile the data, raising time and costs to get the output.
%%
%%An alternative, with low cost, that has been adopted more often to handle scientific problems like this one, is to make use of volunteers for data evaluation, this approach is known as Citizen Science. Volunteers can collaborate with scientific projects, giving away computational or cognitive power.
%%
%%This study will evaluate alternative ways to get the most of the volunteers, using a citizen science project. And will estimate different manners to integrate satellite images acquisition modules, using citizen science as well, so the creation of new areas would ease off, fast and scalable.
%%
%%The organization of this proposal is: chapter \ref{rainforests:sec:intro}, gives a review about the importance of the forests, some current known monitoring systems and their operations. chapter \ref{citizen_science:sec:intro}, introduces citizen science projects, its categories and some of the most succeeded examples. The citizen science project, cited earlier to perform deforestation monitoring, is on chapter \ref{forest_watchers}. At last, the chapter \ref{projectplan:intro} discuss the objectives of this proposal and timetable to be followed.